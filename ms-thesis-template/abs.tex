\vspace{-0.4in}

As a result of the once-through fuel cycle implemented in the US, used nuclear fuel (UNF) steadily increases. One proposed solution is the transition to a closed nuclear fuel cycle, in which reprocessing reduces build up of UNF. Pyroprocessing is an attractive method for this transition for its capabilities separating both LWR and metallic fuels, and inherit proliferation resistance. However, unlike aqueous reprocessing plants, industrial pyroprocessing plants do not yet exist. Similar
to safety-by-design in next generation of reactors, safeguards-by-design should be
a part of reprocessing. Without operational experience, these safeguards-by-design need to be derived through modeling and simulation. 

This thesis develops a medium fidelity generic model, Pyre, capable of simulating a variety of facility configurations. Pyre also allows diversion through a diverter class capable of tracking signatures and observables. Rather than track exact material production we use signatures and observables such as operating temperature, pressure, and current to mimic the capabilities of IAEA inspections.

These capabilities are verified in a transition-scenario of the current US fuel cycle to an SFR-based closed fuel cycle. Key operating parameters are determined through
sensitivity analysis on this scenario, monitoring isotopic changes in material unaccounted for.