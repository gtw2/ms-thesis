\vspace{-0.4in}

As a result of the once-through fuel cycle implemented in the US, used nuclear fuel (UNF) steadily increases. One proposed solution is the transition to a closed nuclear fuel cycle, in which reprocessing reduces build up of UNF. Pyroprocessing is an attractive method for this transition for its capabilities separating both light water reactor (LWR) and metallic fuels, and inherent proliferation resistance. However, unlike aqueous reprocessing plants, industrial pyroprocessing plants do not yet exist. Similar
to safety-by-design in next-generation reactors, reprocessing facilities could be designed with safeguards in mind via safeguards-by-design. Without operational experience, these safeguards-by-design need to be derived through modeling and simulation. 

This thesis develops a medium fidelity generic model, Pyre, capable of simulating a variety of pyroprocessing facility configurations. Pyre also simulates diversion via a diverter class capable of tracking signatures and observables. Rather than only tracking exact material production, we use signatures and observables such as operating temperature, pressure, and current to mimic the capabilities of International Atomic Energy Agency (IAEA) inspections and aid identification of nefarious fuel cycles, or shadow fuel cycles.

These capabilities are verified in a transition scenario of the current US fuel cycle to a sodium fast reactor (SFR) based closed fuel cycle. Key operating parameters are determined through
sensitivity analysis of this scenario, monitoring isotopic changes in material unaccounted for. This work concludes that facility parameters which increase interaction between the salt and waste have more impact on material unaccounted for (MUF). This work also expands the state of the art by exploring the use of sub-facility modeling to increase fuel cycle fidelity.