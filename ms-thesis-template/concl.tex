\chapter{Conclusion}

This thesis was motivated by a lack of medium fidelity pyroprocessing plants in current fuel cycle simulators \cite{borrelli_approaches_2017}. Combined with 
the need for safeguards by design in next generation nuclear fuel cycle facilities, a pyroprocessing facility with diversion capabilities fills a technological gap.
Pyre brings more detailed separations processes to nuclear fuel cycle simulators informed by more limited and specific electrochemical models such as SSPM and AMPYRE \cite{maggos_update_2015}.

\Cyclus provides a modular interface to expand and test the capabilities of reprocessing and material diversion. We developed Pyre in the C++ \Cyclus environment to leverage this
modular framework and test the facility in a key pyroprocessing transition-scenario. We ran a full US fuel cycle transitioning from LWRs to SFRs using only Pyre facilities to facilitate
this transition. We verified Pyre's role in this transition-scenario by observing the simulation's uranium utilization, TRU production, and successful fueling and operation of SFRs to
meet power demands.