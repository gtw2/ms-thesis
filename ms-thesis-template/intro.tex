\chapter[Introduction]{Introduction}
The diversion of significant quantities of \gls{SNM} from the nuclear fuel cycle is a major non-proliferation 
concern \cite{noauthor_serving_2017}. These diversions must be detected in a timely manner using signatures and observables in 
order to properly safeguard the fuel cycle. Timely detection is critical in non-proliferation to discover these shadow fuel cycles
before diverted material is further processed. Pyroprocessing is a used nuclear fuel separations technology for advanced reactors. 
Signatures and observables are used to detect diversion of nuclear material.
The goal of this research is to identify potential signs of material diversion in a pyroprocessing facility and implement models 
of these processes into a detailed pyroprocessing facility archetype to the modular, agent-based fuel cycle simulator, \Cyclus \cite{huff_fundamental_2016}. This facility archetype will equip users of the \Cyclus fuel cycle simulator to investigate 
detection timeliness enabled by measuring signatures and observables in various fuel cycle scenarios.
\section{Motivation}
\subsection{Safeguards}
Currently there are no commercially operated pyroprocessing plants, however various research designs exist in national labs,
notably Argonne National Lab (ANL), Idaho National Lab (INL) and in South Korea, KAERI \cite{michael_f._simpson_developments_2012, lee_advanced, frigo_conceptual_2003}. 
Therefore, prior to construction of any design we 
want to implement safeguards by design. Similar to security by design in next generation reactors, the goal is to include key measurement 
points and access points to the design of the facility. Rather than learn from mistakes, in the future we aim to incorporate safety 
into the design.

\subsection{Pyroprocessing}
For other fuel cycle facilities, we have plenty of operating experience to inform on safeguard construction. For example, with aqueous reprocessing
the International Atomic Energy Agency (IAEA) provides detailed flow-sheets of example facilities \cite{international_atomic_energy_agency_implications_2004}. Multiple modeling
tools have been developed for electrochemical processes such as SSPM and AMPYRE to combat this lack of operational experience for pyroprocessing plants \cite{maggos_update_2015}. 
These tools take a high fidelity approach to model the
chemistry taking place within each chamber. In order to run these tools, the user must have intimate knowledge of the specific facility the flowsheets have
been designed for. There is a gap, however, in the medium fidelity models that can inform on broader fuel cycle applications. A generic facility
capable of modeling changes in operational settings and various layouts has not yet been implemented to a fuel cycle simulator \cite{borrelli_approaches_2017}.


\subsection{Future Fuel Cycles}
As the world begins to consider cleaner forms of energy in response to climate change, nuclear energy has regained traction. A main
concern with nuclear power is the pileup of used nuclear fuel (UNF) as a result of the once-through fuel cycle. 
One suggested solution is converting to a closed fuel cycle \cite{wigeland_nuclear_2014}. There are many approaches to transitioning from our current
fuel cycle to a new or closed cycle. Of these evaluation groups (EGs), those involving sodium fast reactors (SFRs) are of interest. 
Pyroprocessing can transition between current fuel cycle scenarios with light water reactors (LWRs) and SFRs and other metallic fuel.
Therefore, pyroprocessing is under consideration as a means of processing the fuel required to start up new breeder reactors for
the EG01-EG24 transition scenario.


\section{Background}
\subsection{Pyroprocessing}
Pyroprocessing is an electrochemical separation method used primarily for metallic fast reactor fuel.
This reprocessing technique uses molten salt, which differs depending on the facility, to provide a medium for current to travel across.
Molten salt such as LiCl-KCl has a broader stability range compared to water, allowing high potentials to be used for separation.
Traditionally, separation would be conducted in a nitric acid which uses water as its medium.
This becomes a problem when considering heavier elements such as lanthanides and actinides.
Controlling the oxidation states of these elements often requires potentials outside the stability of water.
Hence, pyroprocessing was born to improve non-proliferation and reprocessing capabilities.
\\ \\
In addition to the improved redox control of heavier elements, we co-extract materials of interest so they cannot easily be refined for weapons.
This is done through the electrorefining and electrowinning stages by separating a pure uranium stream as well as a uranium/transuranic (U/TRU) mix stream. 
The U/TRU can then be readily used for fuel fabrication while maintaining proliferation resistance.

\subsubsection{Electrochemical Separations}
Electrochemical separation is the driving force behind pyroprocessing. Electrochemistry relies on the use of Gibbs free energy to determine the required amount of energy to drive a reaction forward.

\begin{figure}[h]
	\centering
	\includegraphics[width=0.8\linewidth]{images/electrochem}
	\caption{Basic example of movement of ions within a galvanic cell \cite{angel}.}
	\label{fig:electrochem}
\end{figure}

Figure \ref{fig:electrochem} demonstrates an electrochemical process that generates electricity as a basic example.
The processes described here follow the same principles but requires energy to run.
As shown in this basic example, ions are exchanged between the anode and cathode in an attempt to balance the potential difference.
In the case of pyroprocessing, the potential difference is artificially applied.
A number of different anodes and cathodes are used to force the desired ions to deliver charge from one end of the cell to the other.
These ions that collect on the surface of the cathode can then be removed from the liquid and separated from the rest of the solution.
By controlling the voltage of the solution as well as the composition of the anode, cathode, and electrolyte we can ensure the removal of unwanted elements/isotopes.


\subsubsection{Voloxidation}
Voloxidation is used following the chopping and decladding of the spent fuel. The process is very similar to annealing in regards to materials. The uranium dioxide is heated to temperatures around 700-1000$^\circ C$ which allows gases and some fission products to escape the fuel pellet, as well as convert UO$_2$ to U$_3$O$_8$ \cite{organisation}. Voloxidation, in most cases, takes place in air which provides plenty of oxygen for oxidization of solid UO$_2$ \cite{jubin_spent_2009}:

\[ 3UO_2 + O_2 \rightarrow U_3O_8 \]

The above reaction is possible because of the expansion of uranium at elevated temperatures. A positive feedback is also established; as the uranium dioxide converts to yellowcake powder, the fuel element expands, exposing more uranium dioxide to oxygen. The rate of this reaction/conversion is dependent on the temperature and gas used. Higher temperatures will yield a faster reaction rate; even ~500 $^\circ C$ is sufficient for 99\% reduction in 4 hours.

An added benefit of running a pyroprocessing voloxidation sub-process at the temperatures previously mentioned, 700-1000$^\circ C$, is the removal of some fission products. The PRIDE facility at KAERI takes it a step further and voloxidates at 1250$^\circ C$ to remove troublesome fission products at the beginning of the cycle\cite{organisation}:

\begin{figure}[h]
	\includegraphics[width=\linewidth]{images/volox_table.png}
	\caption{Voloxidation separation stream composition at 1250 $^\circ C$}
\end{figure}

As shown in the table above, a majority of high activity isotopes are removed from the system at the beginning of pyroprocessing, protecting equipment and workers down the line. These gases are sent to an off-gas treatment facility that makes use of various scrubbing techniques such as liquid scrubbing, cyrogenic distillation (for the krypton), caustic scrubbing, etc \cite{jubin_spent_2009}.

\subsubsection{Electroreduction}
Following off-gassing and conversion to yellowcake, the non-metallic fuel must be converted and reduced to a molten salt mixture. In most cases this is done with a LiCl-KCl salt eutectic combined with Li$_2$O catalyst. The electrolytic reduction phase consists of three main parts: UO$_2$ recovery, reduction, and rare earth (RE) removal.

\begin{figure}[h]
	\centering
	\includegraphics[width=\linewidth]{images/reduction_flow}
	\caption{Electroreduction flow sheet \cite{ohta}.}
\end{figure}

The first step in electrolytic reduction is the recovery of UO$_2$ before reducing the remaining material.
The following equations dictate the transfer of uranium from the anode to cathode.

\[ UO_2 \rightarrow UO_2^{2+}(LiCl-KCl) + 2e^{2-} \hspace{6mm} anode \]
\[ UO^{2+}(LiCl-KCl) + 2e^{2-} \rightarrow UO_2 \hspace{6mm} cathode \]

As in other separations technologies, noble metals can often follow the uranium through the rest of the process.
The lurking noble metal fission products (FPs) cause an increase in radioactivity of the UO$_2$ stream. 
Therefore, the weight percent dissolution of uranium is critical in reducing the amount of waste that follows to the product stream.
Lithium oxide can also be used as a catalyst to draw uranium to the cathode while leaving the noble metal fission products in the salt.
This is done with 1-3wt\% Li$_2$O in the following equations \cite{hur_electrochemical_nodate}:

\[ Li_2O \rightarrow 2Li^+ + O^{2-} \]
\[ UO_{x/2} + xLi \rightarrow U + xLi_2O \]

These equations make a continuously driven loop dragging uranium (either UO$_2$ or U$_3$O$_8$) from the anode to the cathode. 
Disproportionated lithium ions from the first equation break apart the uranium and oxide, with help from the electric potential.
The U collects on the cathode while the Li$_2$O is recycled and drives the first equation forward again. 
Reduction then occurs on the cathode where the U, TRU, REs, and noble metals (NMs) have collected.
This is achieved by evolving oxygen gas along the anode using the following reactions\cite{hur_electrochemical_nodate,organisation}:

\[ Li^+ \rightarrow Li + e^- \hspace{10mm} Cathode \]
\[ M_xO_y + 2yLi \rightarrow xM + yLi_2O \hspace{10mm} Cathode \]
\[ O^{2-} \rightarrow 0.5O_2 + 2e^- \hspace{10mm} Anode \]

Electrochemical reduction results in an alloy of reduced U/TRU/RE/NM; however, we want to minimize the amount of RE and NM in the product.
We've touched already on how to reduce the quantity of NM and for the final step the RE must be removed.
The RE FPs can be removed from the alloy by substituting another chloride into the LiCl-KCl eutectic.
In the case of Ohta et al. ZrCl$_4$ was considered \cite{ohta}:

\[ 3ZrCl_4(LiCl-KCl) + RE \rightarrow 3Zr + 4RECl_3(LiCl-KCl) \]

This process is shown to have a decontamination factor of 10 in regards to separating REs from actinides \cite{sakamura}. Additionally, by using Zr as the metal substitute, it is compatible with fuel fabrication later \cite{ohta}.
\subsubsection{Electrorefining}
Electrorefining is the primary process in pyroprocessing, and is the feed point for fast reactor fuel, since it does not require reduction or chopping.
Being the most important process, it is also the most complex with a multitude of input parameters and outputs. 
The goal of the refining process is to separate the uranium and TRU from the alloy ingot formed in the reduction phase.
Two streams will be formed for the fabrication of fuel: one stream that is a mix of U/TRU at the desired ratio, and the other a pure stream of uranium.
The refining step's efficiency relies on temperature and current primarily, however, advanced methods are being developed.
KAERI for example has conducted work on adding a central stirrer, lowering pressure, and rotating the anode \cite{lee_advanced}.
The rotation aims to mix the uranium in the salt such that none gets stuck on the bottom or edges of the vessel. 
Stirring too vigorously, however, can lead to the removal of uranium dendrites from the cathode thereby decreasing efficiency.\\

The governing reactions that allow this process to work are based on the stability constants and oxidation potential of the remaining FPs.
The voltage used,  0.5-1V, is such that Uranium is unstable in the chloride form \cite{organisation}, while TRUs have a higher stability. 
This leads to TRU remaining in chloride form, along with some uranium, and pure uranium accumulating on the cathode.
The chloride reaction follows the below equation, and will run to the right as long as there is uranium within the salt \cite{organisation}.

\[ UCl_3+TRU(RE) \rightarrow U + TRU(RE)Cl_3 \]
\[ UCl_3 + 3Na \rightarrow 3NaCl + U \]
\[ UCl_3 + 3Cs \rightarrow 3CsCl + U \]
\[ UCl_3 + Pu \rightarrow PuCl_3 + U \hspace{10mm} \delta G = -22.44kcal \]
\[ 4UCl_3 + 3Zr \rightarrow 3ZrCl_4 + 5U \hspace{10mm} \delta G = 31.123kcal \]

As shown by the reactions above, the TRU have a favorable Gibbs free energy value for spontaneous reactions while the transition metals do not \cite{supy}.
This leads to the transition metals remaining in the anode basket while the TRU are drawn into the liquid cadmium cathode \cite{lee_korean_2011}.


\subsubsection{Electrowinning}
The electrorefiner accumulates TRUs and rare earth fission products within the salt.
These isotopes build up and require separation and disposal, therefore the salt from the refiner is sent to the electrowinner.
This stage further purifies the salt by targeting the electric potential of TRUs, RE, and U again \cite{lee_korean_2011,organisation}.
Placed in liquid cadmium once again, the three groups have overlapping electric potentials leading them all to deposit in the cadmium \cite{lee_korean_2011}. 
While the refiner's role is to generate a stream of pure uranium, the electrowinner performs co-extraction of Uranium and TRUs.
This inherent proliferation resistance is a main draw of the pyroprocessing technique.
Rare earths are still present on the cadmium and further separations must be conducted.
These elements are removed through the addition of CdCl$_2$ which oxidizes the rare earths, while the uranium and TRUs are unaffected.
These oxidized elements fall back into the salt, leaving the purified U/TRU stream on the electrowinner.


Although the facility is great in terms of safeguards, pyroprocessing has its share of drawbacks as well.
Currently, pyroprocessing can only be performed as a batch process, which significantly limits throughput compared to a continuous facility. 
Additionally, the safety and economic concerns of running a molten salt plant are much greater than a nitric acid one.
Despite these downsides, pyroprocessing is an efficient use of electrochemical separation and a leader in proliferation resistant separations.

There are multiple different designs for a pyroprocessing facility, the most prominent being from ANL, INL, and KAERI. In order to encompass them all, we must take a generic approach. This is accomplished by including the following sub-processes: voloxidation, electroreduction, electrorefining, and electrowinning. While electrorefining is the process of primary concern, each of the processes has an important in role in various processing plants. 

\section{Goals}

The goals of this work are to appropriately model a generic pyroprocessing facility with medium fidelity capable of diversion. With this model
in \Cyclus we wish to explore the capability of modeling sub-facilities and diversion. In addition, we will use this higher fidelity model to verify transition
scenarios such as EG01-EG24 within \Cyclus \cite{wigeland_nuclear_2014}. Finally we wish to evaluate optimum detector placement and measurement points for
various facility layouts through sensitivity analysis. 